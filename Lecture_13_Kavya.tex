\documentclass{article}
\usepackage[utf8]{inputenc}
\usepackage[a4paper,margin=3.5cm]{geometry} %Sets the page geometry
\usepackage{url}
\usepackage{dirtytalk}
\usepackage{graphicx} % Package for \includegraphics
\usepackage{wrapfig} % Figure wrapping
\usepackage[T1]{fontenc} % Output font encoding for international characters
\setlength{\parskip}{1em} % Set space when paragraphs are used
\usepackage{amssymb}
\usepackage{amsmath}
\usepackage{tcolorbox}
\usepackage{mathtools}
\usepackage{tikz}
\usepackage{amsthm}
\usepackage{caption}
\usepackage{changepage}

\title{Lecture 13}
\author{Kavya Gupta}
\date{September 2023}

\begin{document}
\maketitle
This lecture was about Relations and Functions, something closely related with the Sets, covered in previous lecture.\\ \\
\textbf{Definition :} A relation from set $A$ to set $B$ is a subset $R$ of $A \times B$ collections of pairs $(a, b)$ such that $a \in A, b \in B$ if $(a, b) \in R$, we say $a$ is related to $b$ by $R$..
\[aRb \iff (a, b) \in R\]
Note : P(x, y) is a  predicate where $x \in A$ and $y \in B$.\\ \\
A relation on a set $A$ is a relation from $A$ to $A$.\\ \\
\textbf{Definition :} A function from $A$ to $B$ is a relation from $A$ tot $B$ that satisfies $\Rightarrow$
\[(\forall a \in A \: \exists \: b \in B \;  \ni \; (a, b) \in R) \bigwedge (\forall a \in A \; \forall \, b_{1}, \, b_{2} \in B, \; (a_{1}, b_{1}) \in R \; \wedge \; (a_{1}, b_{2}) \in R \Longrightarrow (b_{1} = b_{2})) \]
A function follows more properties in general than relation, every element is mapped to exactly one element.\\ \\
Note : \textsc{A function is also a relation.}\\
If $f$ is a function then $b=f(a)$ denotes that $(a, b) \in f$.\\ \\
\textbf{One to One function :} A function from $A$ to $B$ is 1-1 (one to one) if..
\[\forall a_{1}, a_{2} \in A : f(a_{1}) = f(a_{2}) \Longrightarrow a_{1} = a_{2}\]
\textbf{Onto Function :} A function from $A$ to $B$ is onto if $\forall b \in B \; \exists \; a \in A \ni f(a)=b$\\ \\
A function both 1-1 and onto is called a \textit{bijection}.
\subsubsection*{\textsc{Theorem}}
There cannot exist a bijection from $A$ to $2^{A}$, where $2^{A}$ denotes the power set of $A$.\\ \\
Note : Intuitively, it feels that $|A| < |2^{A}|$, hence bijection can't exist. But that's only for \textbf{finite} sets, and we can't comment on infinite sets like $\mathbb{N}$ like this. \\ \\
\textbf{Proof :}\\
Proof by Contradiction.\\
Suppose there exists a bijection $f$ from $A$ to $2^{A}$.\\ \\
Consider subset $x$ of \textbf{ALL} the elements in $A$ such that $x \notin f(x)$, call this set $B$.\\ \\
Since $f$ is onto, then $\exists$ an element in $b \in A$ such that $f(b) = B$.
\\
\\
\textbf{Important Note :} This $B$ is a collection of subsets of $A$. Hence $B$ is itself a subset of $A$, which will be contained in $2^{A}$. So, for some $b \in A$, there is $B \in 2^{A}$.
\\
\\
\textbf{Is $b \in B$ true ?}
\\
If this is true then $b \notin f(b)$ but this implies $b \notin B$ as $f(b) = B$. Contradiction.\\
If this is false then $b \in f(b)$ as $B$ consists of \textbf{ALL} of the elements such that $x \notin f(x)$ hence $B^{c}$ consists of all points such that $x \in f(x)$, hence again a contradiction. \\ \\
Note : Here it doesn't matter whether set $B$ will be empty or not. What's important is that $B \in 2^{A}$ whatever $B$ be.\\ \\
Hence our assumption of bijection fails.\\ \\
Note : This problem was only solved using the \textit{surjection}. We didn't use the one-to-one thing.
\subsubsection*{\textsc{Question}}
Does there exist a bijection between $\mathbb{N}$ and $\mathbb{N} \times \mathbb{N}$ ? \\ \\
\textbf{Answer :} Yes !\\ \\
Basically idea is to traverse diagonally over the lattice points and label them as $0 \Rightarrow 1 \Rightarrow 2 \Rightarrow 3 \Rightarrow...$.
\\ \\
Idea to admire that though intuitively it feels like $|\mathbb{N}| << |\mathbb{N} \times \mathbb{N}|$ and hence they should be not bijective sets but still we are able to find a bijection.\\ \\
This is nothing but the magic of \textit{infinity} $\infty$... \\ \\
Infinity is an idea, not a number. Infinity has it own definitions and shapes. Here the infinity of $\mathbb{N} \times \mathbb{N}$ is larger than that of $\mathbb{N}$...\\ \\
Still as both the sets are infinite, our intuitive thoughts may fail. \\ \\
It so happens that there exists a bijection even for $\mathbb{N}$ and $\mathbb{N} \times \mathbb{N} \times \mathbb{N} \times ...\; k$ times, $k$ is finite.\\ \\
As $k$ tends to $\infty$, the set becomes $\mathbb{N}^{\mathbb{N}} = \mathbb{N} \times \mathbb{N} \times \mathbb{N} \times ... \; \infty$, then there doesn't exist a bijection of it with $\mathbb{N}$.\\ \\
\textbf{Note : Diagram on next page.}
\newpage
\subsubsection*{Further}
\begin{figure}
    \centering
    \includegraphics[width = 0.5\textwidth]{Pic.png}
    \caption{Bijection from $\mathbb{N}$ to $\mathbb{N} \times \mathbb{N}$}
\end{figure}
\subsubsection*{\textsc{Another Question}}
Does there exist a set $S$ such that there are 1-1 functions from $\mathbb{N}$ to $S$ and from $S$ to $2^{\mathbb{N}}$ but no bijection from $\mathbb{N}$ to $S$ and from $S$ to $2^{\mathbb{N}}$ ?
\[|\mathbb{N}| < |S| < |2^{\mathbb{N}}|\]
\textbf{Comment :} There has been nothing to prove as well as to disprove this statement... Hence it still remains a mystery and we can do nothing but give our hypothesis on it.\\ \\
\textbf{Continuum Hypothesis :} There exists no such set $S$ between $\mathbb{N}$ and $\mathbb{N} \times \mathbb{N}$. \\ \\
Considering or not considering this hypothesis, creates two different Set theory, independent of fundamental axioms of sets.\\
But at some point, we would reach disagreement between these two set theories for a particular set, with different properties in either.
\end{document}